% !TEX TS-program = pdflatex
% !TEX encoding = UTF-8 Unicode

% This is a simple template for a LaTeX document using the "article" class.
% See "book", "report", "letter" for other types of document.

\documentclass[11pt]{article} % use larger type; default would be 10pt

\usepackage{graphicx} % support the \includegraphics command and options

% \usepackage[parfill]{parskip} % Activate to begin paragraphs with an empty line rather than an indent

%%% PACKAGES
\usepackage{subfig} % make it possible to include more than one captioned figure/table in a single float
% These packages are all incorporated in the memoir class to one degree or another...
\usepackage{amsmath}
% To use multi-column and multi-row cells in tables
\usepackage{multirow}
% to get figures on the right spot (capital H parameter)
\usepackage{float}
% set of complex numbers
\usepackage{amssymb}
% makes sure the paragraphs are seperated with an enter space
\usepackage[parfill]{parskip}

%%% The "real" document content comes below...

\title{Fourier Transformatie en benaderingstheorie}
\author{Willem Melis}
%\date{} % Activate to display a given date or no date (if empty),
         % otherwise the current date is printed

\begin{document}
\maketitle

\section{Projection of vectors}
This text assumes that the reader has already had a first experience with simple vector algebra. 

Matrix $A$ is an normal (n,m) matrix . So columns of A [$a_1,a_2 ... a_n$] are orthogonal(and also normal and linear in-depended) towards each other. This means that any vector V can be approximated with the column space of $A$.

The dot product between two vectors is defined  as: $dot(f,g) = f^Tg$ The orthogonal project of a vector on one of the 

\begin{eqnarray}
v =  \hat{v} + residue\\
\hat{v} = \sum_i (a_i^Tv)a_i
\end{eqnarray}
	
The v hat is the approximation of v in the mean square sense. If  $v$ is in  $span\{a_1 a_2 .. a_n\}$ then $\hat{f} =f$ and so the residual will be zero.

\section{Projection of functions}

Dezelfde redenering van bij vectoren werkt ook op functies, we een functieruimte G met orthonormale functies $g_1 ... g_n$. Dan kunnen we dus de functie f benadering in deze functie ruimte.


\begin{equation}
	dot(f,g) =  \int_{-\infty}^{\infty} f(t) g(t) w(t) dt
\end{equation}

En nu is onze benadering

\begin{equation}
\hat{f}(t) =  \sum_k [\int_{-\infty}^{\infty} f(t) g_k(t) w(t) dt]g_k(t)
\end{equation}

of

\begin{eqnarray}
\lambda_i =   \int_{-\infty}^{\infty} f(t) g_i(t) w(t)dt \\
\hat{f} = \sum_i \lambda_i g_i(t)
\end{eqnarray}

Als de functieruimte groot genoeg gekozen is zal $f=\hat{f}$ en dus zal $f = \sum_i \lambda_i g_i(t)$. 

\section{Fourier}
Sommige lezers zullen al door hebben dat wat we voordien deden met een normale(en orthogonale) functie dus geld voor de functie ruimte van de Fourier transformatie. We stellen de gewichtsfunctie gelijk aan $w(t)=1$. De functie f(t) is de functie die we benaderen.

\begin{eqnarray}
	g_k=e^{-2\pi i t k} \\
	\hat{f}(t) =  \sum_k [\int_{-\infty}^{\infty} f(t) g_k(t) dt] g_k(t)
\end{eqnarray}

We kunnen dit verder uitbreiden naar:
\begin{eqnarray}
h(k) =  \int_{-\infty}^{\infty} f(t) e^{-2\pi i t k} dt \\
f(t) = \hat{f}(t) = \int_{-\infty}^{\infty} h(k) e^{-2\pi i t k} dk 
\end{eqnarray}

K is in dit geval de frequentie(integratie met substitutie en de typische $2 \pi$ term komt tevoorschijn), en t is de tijd.

Wat zou er gebeuren moest $w(t)$ een dalende exponentiële zijn? Wel dan komen we de formules van Laplace uit.

\end{document}
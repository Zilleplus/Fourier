% !TEX TS-program = pdflatex
% !TEX encoding = UTF-8 Unicode

% This is a simple template for a LaTeX document using the "article" class.
% See "book", "report", "letter" for other types of document.

\documentclass[11pt]{article} % use larger type; default would be 10pt
\usepackage[dutch]{babel}

\usepackage[utf8]{inputenc} % set input encoding (not needed with XeLaTeX)

%%% Examples of Article customizations
% These packages are optional, depending whether you want the features they provide.
% See the LaTeX Companion or other references for full information.

%%% PAGE DIMENSIONS
\usepackage{geometry} % to change the page dimensions
\geometry{a4paper} % or letterpaper (US) or a5paper or....
\geometry{margin=1in} % for example, change the margins to 2 inches all round
% \geometry{landscape} % set up the page for landscape
%   read geometry.pdf for detailed page layout information

\usepackage{graphicx} % support the \includegraphics command and options

% \usepackage[parfill]{parskip} % Activate to begin paragraphs with an empty line rather than an indent

%%% PACKAGES
\usepackage{booktabs} % for much better looking tables
\usepackage{array} % for better arrays (eg matrices) in maths
\usepackage{paralist} % very flexible & customisable lists (eg. enumerate/itemize, etc.)
\usepackage{verbatim} % adds environment for commenting out blocks of text & for better verbatim
\usepackage{subfig} % make it possible to include more than one captioned figure/table in a single float
% These packages are all incorporated in the memoir class to one degree or another...
\usepackage{amsmath}
% To use multi-column and multi-row cells in tables
\usepackage{multirow}
% Include source code
\usepackage{listings}
% to get figures on the right spot (capital H parameter)
\usepackage{float}
% set of complex numbers
\usepackage{amssymb}
% makes sure the paragraphs are seperated with an enter space
\usepackage[parfill]{parskip}
% get a decent way to use quotes
\usepackage{csquotes}


%%% HEADERS & FOOTERS
\usepackage{fancyhdr} % This should be set AFTER setting up the page geometry
\pagestyle{fancy} % options: empty , plain , fancy
\renewcommand{\headrulewidth}{0pt} % customise the layout...
\lhead{}\chead{}\rhead{}
\lfoot{}\cfoot{\thepage}\rfoot{}

%%% SECTION TITLE APPEARANCE
\usepackage{sectsty}
\allsectionsfont{\sffamily\mdseries\upshape} % (See the fntguide.pdf for font help)
% (This matches ConTeXt defaults)

%%% ToC (table of contents) APPEARANCE
\usepackage[nottoc,notlof,notlot]{tocbibind} % Put the bibliography in the ToC
\usepackage[titles,subfigure]{tocloft} % Alter the style of the Table of Contents
\renewcommand{\cftsecfont}{\rmfamily\mdseries\upshape}
\renewcommand{\cftsecpagefont}{\rmfamily\mdseries\upshape} % No bold!

%%% END Article customizations

%%% The "real" document content comes below...

\title{Fourier Transformatie en benaderingstheorie}
\author{Willem Melis}
%\date{} % Activate to display a given date or no date (if empty),
         % otherwise the current date is printed

\selectlanguage{dutch}

\begin{document}
\maketitle

\section{Projectie van vectoren}
De lezer zou al moeten bekend zijn met het projecteren van vectoren:

Matrix $A$ is een normale n*m matrix . De kolommen ($a_1,a_2 ... a_n$) zijn allemaal orthogonaal t.o.v. elkaar. Dit betekend dat een vector V kan benaderd worden door de kolomruimte van $A$.

We definiëren het dotproduct als $dot(f,g) = f^Tg$

\begin{eqnarray}
v =  \hat{v} + residu\\
\hat{v} = \sum_i (a_i^Tv)a_i
\end{eqnarray}
	
Deze benadering zal de kwadratische fout minimaliseren. Als $v$ zit in de $span\{a_1 a_2 .. a_n\}$ dan is de $\hat{f} =f$ en de residu dus 0.

\section{Projectie van functies}

Dezelfde redenering van bij vectoren werkt ook op functies, we een functieruimte G met orthonormale functies $g_1 ... g_n$. Dan kunnen we dus de functie f benadering in deze functie ruimte.


\begin{equation}
	dot(f,g) =  \int_{-\infty}^{\infty} f(t) g(t) w(t) dt
\end{equation}

En nu is onze benadering

\begin{equation}
\hat{f}(t) =  \sum_k [\int_{-\infty}^{\infty} f(t) g_k(t) w(t) dt]g_k(t)
\end{equation}

of

\begin{eqnarray}
\lambda_i =   \int_{-\infty}^{\infty} f(t) g_i(t) w(t)dt \\
\hat{f} = \sum_i \lambda_i g_i(t)
\end{eqnarray}

Als de functieruimte groot genoeg gekozen is zal $f=\hat{f}$ en dus zal $f = \sum_i \lambda_i g_i(t)$. 

\section{Fourier}
Sommige lezers zullen al door hebben dat wat we voordien deden met een normale(en orthogonale) functie dus geld voor de functie ruimte van de Fourier transformatie. We stellen de gewichtsfunctie gelijk aan $w(t)=1$. De functie f(t) is de functie die we benaderen.

\begin{eqnarray}
	g_k=e^{-2\pi i t k} \\
	\hat{f}(t) =  \sum_k [\int_{-\infty}^{\infty} f(t) g_k(t) dt] g_k(t)
\end{eqnarray}

We kunnen dit verder uitbreiden naar:
\begin{eqnarray}
h(k) =  \int_{-\infty}^{\infty} f(t) e^{-2\pi i t k} dt \\
f(t) = \hat{f}(t) = \int_{-\infty}^{\infty} h(k) e^{-2\pi i t k} dk 
\end{eqnarray}

K is in dit geval de frequentie(integratie met substitutie en de typische $2 \pi$ term komt tevoorschijn), en t is de tijd.

Wat zou er gebeuren moest $w(t)$ een dalende exponentiële zijn? Wel dan komen we de formules van Laplace uit.

\end{document}